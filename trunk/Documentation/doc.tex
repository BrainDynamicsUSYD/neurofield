\documentclass[12pt,a4paper]{article} 

\title{NeuroField Reference}
\author{Peter Drysdale \and Felix Fung \and Romesh Abeysuriya}
\date{\today}

\usepackage[latin1]{inputenc}
\usepackage{tikz}
\usetikzlibrary{shapes,arrows}

\usepackage{listings}
\usepackage{hyperref}
\usepackage{graphicx}
\usepackage[top=2cm,bottom=2cm,left=1.3cm,right=1.3cm]{geometry}
\usepackage{multirow}
\usepackage{amsmath}

\pagestyle{plain}
\parindent=0.0cm
\parskip=0.5cm

\footskip=1.3cm

\lstset{
basicstyle=\small\small\ttfamily,
frame=single,	        % adds a frame around the code
breaklines=true,		% sets automatic line breaking
breakatwhitespace=tru,	% sets if automatic breaks should only happen at whitespace
}
\newcommand{\code}[1]{ 
\begin{lstlisting}
#1
\end{lstlisting}
}
\newcommand{\type}[1]{ {\small\small\tt #1} }

\begin{document}

\maketitle
\abstract{\type{NeuroField} is a computer program (accompanied with helper scripts) that solves the neural field model of Robinson et al. This document is a reference of \type{NeuroField} for both users and developers.}
\tableofcontents

\pagebreak
\section{Users guide}

\type{NeuroField}, written by Peter Drysdale with contributions from James Roberts, Felix Fung and Romesh Abeysuriya, is a \type{C++} program (accompanied with helper scripts) that solves the neural field model of Robinson et al.:
\begin{align*}
	D_{ab}V_{ab}(\mathbf{r},t) &= \nu_{ab}\phi_{ab}(\mathbf{r},t),\\
			Q_a(\mathbf{r},t) &= S_a \big[\sum_b V_{ab}(\mathbf{r},t) \big],\\
	\mathcal{D}_{ab}\phi_{ab}(\mathbf{r},t) &= Q_b(\mathbf{r},t-\tau_{ab}).
\end{align*}
\type{NeuroField} generalizes the neural field theory by allowing users to:
\begin{enumerate}
	\item Specify an arbitrary number of populations and connections between populations;
	\item Specify the parameters for any objects, including populations, dendritic responses, firing responses, propagators, synapses, and stimulus pattern;
	\item Choose alternative wave propagation types, i.e. choose different forms of \(\mathcal{D}_{ab}\);
	\item Uses plastic synapses, i.e. \(\nu_{ab}=\nu_{ab}(\mathbf{r},t)\).
\end{enumerate}

This users guide covers the obtaining and setting up (Sec.~\ref{sec:obtain}), configuring (Sec.~\ref{sec:config}) and launching of \type{NeuroField} (Sec.~\ref{sec:launch}).

Within this documentation, specific terminology as appeared in the computer is in \type{typwriter font}. Commands are denoted as
\begin{lstlisting}
Command to put in computer
\end{lstlisting}

\subsection{Obtaining and setting up NeuroField}
\label{sec:obtain}

The code for \type{NeuroField} is managed by a version control system called \type{subversion}, which provides a single place to obtain the latest copy of the code, as well as storing the entire history of the program. To access the repository, contact Romesh Abeysuriya (\url{r.abeysuriya@physics.usyd.edu.au}) or Sue Yang (\url{xue.yang@sydney.edu.au}).

To set up the latest version of \type{NeuroField} in the current directory within the School of Physics, execute
\begin{lstlisting}
svn co http://silliac.physics.usyd.edu.au:18080/svn/neurofield/trunk neurofield --username=<your SVN username>
\end{lstlisting}

You can also use these steps to obtain a copy of \type{NeuroField} from a computer which is not connected to the School of Physics network (eg. personal laptops at home). You will require a version of \type{SVN} higher than 1.6. However, you will need to have your IP address/network domain registered for remote access, please contact Sebastian Juraszek to request this (ideally by logging a helpdesk request at \url{http://physics.usyd.edu.au/itsupport} - only available within the School of Physics). 

\subsection{Directory layout}

The canonical directory is \type{neurofield/trunk}. Within this directory, the user can find:

\begin{tabular}{l p{12cm}}
\type{*.h, *.cpp}& C++ source code.\\
\type{Configs/}& Stores configuration files for \type{NeuroField}.\\
\type{Documentation/}& \type{Documentation/doc.tex} is the \LaTeX\ file for generating this document. Running \type{make doc} produces this document in \type{pdf} form